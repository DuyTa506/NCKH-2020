\newpage
\centerline{\bf THÔNG TIN KẾT QUẢ NGHIÊN CỨU CỦA ĐỀ TÀI}

\noindent 
{\bf 1. Thông tin chung:}

- Tên đề tài: \blue{Phân tích chuỗi thời gian bằng mô hình ARIMA với phần mềm R}

- Mã số: \textbf{S2019.564.01}

- Sinh viên thực hiện: - Cao Thị Ái Loan$^{1}$
                       
            \hspace{4cm}- Phùng Thị Hồng Diễm$^{1}$
            
            \hspace{4cm}- Nguyễn Quốc Dương$^{2}$
                       
            \hspace{4cm}- Lê Phương Thảo$^{2}$
            
            \hspace{4cm}- Đinh Thị Quỳnh Như$^{2}$
            
$^1$Lớp Toán học K39 Khoa: Toán và Thống kê\quad Năm thứ: 4\quad Số năm đào tạo: 4

$^2$Lớp Sư Phạm Toán K40 \quad Khoa: Sư Phạm \quad Năm thứ: 3\quad Số năm đào tạo: 4

- Người hướng dẫn: TS. Lê Thanh Bính

\noindent 
{\bf 2. Mục tiêu đề tài:}

- Nắm vững lý thuyết cơ bản về chuỗi thời gian và mô hình ARIMA; xây dựng quy trình dự báo bằng mô hình ARIMA gồm 5 bước sau: kiểm tra tính dừng của chuỗi thời gian; chuyển một chuỗi không dừng thành dừng; xác định bậc $p$, $q$, $P$, $Q$; kiểm tra độ chính xác của mô hình; bước cuối cùng là, dự báo.

- Nắm vững quy trình xây dựng mô hình ARIMA (với 5 bước nêu trên) bằng phần mềm R cho các dữ liệu thời gian thực trong những lĩnh vực khác nhau như khí tượng thủy văn; dịch tễ học; giáo dục; tài chính; chứng khoán.

\noindent 
{\bf 3. Tính mới và sáng tạo:} Áp dụng mô hình ARIMA để nghiên cứu độc lập trên các tập dữ liệu thực tế, ứng dụng trực tiếp cho địa phương và dự báo dịch bệnh COVID-19. Hơn nữa, chúng tôi so sánh khả năng dự báo của mô hình ARIMA với NNAR để có cái nhìn khách quan hơn.

\noindent 
{\bf 4. Kết quả nghiên cứu:}
Phân tích tổng quan tình hình COVID-19 trên toàn thế giới, phân tích và dự báo số ca nhiễm mới COVID-19 tại Mỹ và số ca tử vong mới COVID-19 tại Italy, dự báo lượng mưa tại trạm quan trắc Quy Nhơn và tạo một website Dashboard COVID-19 để theo dõi xu hướng dịch bệnh bằng mô hình ARIMA.
 
\noindent 
{\bf 5. Đóng góp về mặt kinh tế-xã hội, giáo dục và đào tạo, an ninh, 
quốc phòng và khả năng áp dụng của đề tài:}
Đề tài hoàn thành là tài liệu tham khảo hữu ích cho những ai quan tâm đến phân tích dữ liệu chuỗi thời gian, đặc biệt là mô hình ARIMA với phần mềm R. Hơn nữa, đề tài còn góp phần vào công cuộc chống đại dịch COVID-19 bằng cách xây dựng website Dashboard COVID-19.

\noindent 
{\bf 6. Công bố khoa học của sinh viên từ kết quả nghiên cứu của đề tài:}

- Bài báo "\textbf{Monthly Rainfall Forecast of Quy Nhon using SARIMA Model}" được chấp nhận đăng trên \textit{Tạp chí khoa học Trường đại học Quy Nhơn}.

- Bài báo "\textbf{Predicting the Pandemic COVID-19 using ARIMA Model}" được chấp nhận đăng trên tạp chí \textit{VNU Journal of Science: Mathematics-Physics}, Đại học Quốc gia Hà Nội (tạp chí được xuất bản bằng tiếng Anh).

- Đã gửi bản thảo bài báo "\textbf{Modeling Total Vehicle Sales data in USA to forecasting: A comparison between the Holt-Winters, ARIMA and NNAR models}" đến tạp chí \textit{International Journal of Applied Mathematics and Statistics} (India).

\hfill Ngày  28 tháng 05 năm 2020 \mbox{\qquad}

\hfill {\bf Sinh viên chịu trách nhiệm chính}

\hfill {\bf thực hiện đề tài} \mbox{\qquad \qquad}
\\
\\

\hfill {\bf Cao Thị Ái Loan}\mbox{\qquad\;\;\;\;\;\;\;}
\\

\noindent
{\bf Nhận xét của người hướng dẫn về những đóng góp khoa học của sinh viên thực hiện đề tài:}

Nhóm sinh viên thực hiện đề tài đã dành rất nhiều thời gian, công sức để tập trung tìm tòi và nỗ lực đọc hiểu kĩ các tài liệu bằng tiếng Anh với lượng kiến thức rất lớn liên quan đến xác suất thống kê, lý thuyết chuỗi thời gian, mô hình ARIMA và các công cụ của phần mềm R. Trong quá trình thực hiện nghiên cứu, có những vấn đề nảy sinh không có trong tài liệu chuyên khảo hoặc đề cập nhưng không rõ ràng, nhóm tác giả đã chủ động trao đổi trên diễn đàn thống kê học máy với các chuyên gia và đã học hỏi được nhiều kiến thức. Tôi đánh giá rất cao phương pháp và tinh thần chủ động học tập đó. Hơn nữa, nhóm sinh viên đã giành rất nhiều thời gian để nghiên cứu và học cách trình bày kết quả nghiên cứu dưới dạng bài báo bằng tiếng Anh. Chính vì vậy, tôi dành lời khen rất lớn về thái độ, tinh thần làm việc, say mê nghiên cứu của nhóm sinh viên. Một lần nữa, tôi cho rằng nhóm sinh viên thực hiện đề tài đã hoàn thành rất xuất sắc công việc mà người hướng dẫn đã đặt ra ban đầu.

\hfill Ngày 28 tháng 05 năm 2020

\hfill{\bf Xác nhận của Khoa \hskip6cm  Người hướng dẫn} \mbox{\quad}
\\
\\
\\

\hfill{\bf PGS. TS. Lê Công Trình \hskip5cm  TS. Lê Thanh Bính \;} \mbox{\quad}